\documentclass[dcc]{fcfmcourse}
\usepackage{teoria}
\usepackage[utf8x]{inputenc}
\usepackage{amsmath}
\usepackage{amsfonts,setspace}
\usepackage{listings}
\usepackage{color}
\usepackage{xcolor}
\usepackage{listings}
\lstset{basicstyle=\ttfamily,
  showstringspaces=false,
  commentstyle=\color{red},
  keywordstyle=\color{blue}
}
\usepackage{cancel}
\usepackage{hyperref}
\usepackage{epstopdf}
\usepackage{fancyhdr}
\pagestyle{fancy}
%\cfoot{``In mathematics the art of asking questions is more valuable than solving problems" \\ Georg Cantor}
\definecolor{pblue}{rgb}{0.13,0.13,1}
\definecolor{pgreen}{rgb}{0,0.5,0}
\definecolor{porange}{rgb}{0.9,0.5,0}
\definecolor{pgrey}{rgb}{0.46,0.45,0.48}

\lstset{language=Java,
  showspaces=false,
  showtabs=false,
  breaklines=true,
  showstringspaces=false,
  breakatwhitespace=true,
  commentstyle=\color{porange},
  keywordstyle=\color{pblue},
  stringstyle=\color{pgreen},
  basicstyle=\ttfamily,
  moredelim=[il][\textcolor{pgrey}]{$ $},
  moredelim=[is][\textcolor{pgrey}]{\%\%}{\%\%}
}

\newenvironment{codebox} {\small \ttfamily \obeylines \begingroup \setstretch{-2.4}} {\endgroup}

\title{Tarea 1}
\course[CC3102]{Teoría de la Computación}
\professor{Gonzalo Navarro}
\assistant{Manuel Cáceres}
\assistant{Ian Letter}

% Si pasas el comando usedate a la clase, la fecha aparecerá bajo la lista de auxiliares.
% Puedes usar el formato de fecha por defecto de latex (y traducirla usando babel)
% o puedes escribir lo que quieras con el comando \date.
% \date{1 de Septiembre, 2015}

\begin{document}
\maketitle
\begin{center}
Fecha de entrega : 21 de Octubre del 2016
\end{center}
\vspace{-1ex}

\section*{Descripción}
El objetivo de esta tarea es construir una versión simplificada de la herramienta Linux ``grep''\footnote{Para más información de ``grep'', puede consultar su manual en \url{http://www.gnu.org/software/grep/manual/grep.html}}.\\
La versión más simple de ``grep'' permite buscar un patrón o expresión regular en las líneas de la entrada estándar. Cuando ``grep'' encuentra el patrón dentro de una línea simplemente imprime la línea completa a la salida estándar, es decir, ``grep'' busca substrings de la línea que estén en el lenguaje definido por la expresión regular correspondiente y si encuentra alguno imprime esta línea.\\

Para esto considere:
\begin{enumerate}[a)]
    \item Construir un AFND $N$ que reconozca la expresión regular, según el procedimiento descrito en clases.
    \item Agregar loops al estado inicial de $N$ para buscar en texto.
    \item Convertir el AFND $N$ a un AFD $D$, según el procedimiento descrito en clases.
    \item Correr $D$ en cada una de las líneas de la entrada estándar para averiguar si el patrón se encuentra en esta línea.
    \item Imprimir, a la salida estándar, en caso que corresponda.
\end{enumerate}

\section*{Notación}
El texto está construído exclusivamente con símbolos del alfabeto $\Sigma = \{a,\ldots,z,A,\ldots,Z,0,\ldots,9,\ \}$ (alfanuméricos y espacio). De igual manera, las expresiones regulares se construyen sobre el mismo alfabeto $\Sigma$ y en notación infija.\\

Para facilitar el parsing, la definición de expresiones regulares que consideraremos es la siguiente:
\begin{enumerate}
    \item Para todo $c \in \Sigma$, $c$ es una expresión regular
    \item Sean $e_{1}$ y $e_{2}$ expresiones regulares, entonces
    \begin{enumerate}
        \item $(e_{1}|e_{2})$ es la expresión regular para la unión
        \item $(e_{1}.e_{2})$ es la expresión regular para la concatenación
        \item $(e_{1}*)$ es la expresión regular para la clausura de Kleene
    \end{enumerate}
\end{enumerate}
Las cuales tienen los significados vistos en clases.\\

Observe que una manera simple de parsear la expresión regular consiste en utilizar una adaptación del algoritmo de ``dos pilas'' de Dijkstra\footnote{Una explicación breve y didáctica del algoritmo se puede encontrar en \url{http://algs4.cs.princeton.edu/lectures/13StacksAndQueues.pdf}}.
\section*{Entrega}
\begin{itemize}
    \item El plazo de entrega vence el día 21 de Octubre del 2016. La entrega debe ser en grupos de hasta 3 personas. Debe implementar su programa en C, C++ o Java.
    \item El programa debe recibir un argumento que corresponde a la expresión regular que se buscará. Además lee de la entrada estándar un texto e imprime aquellas líneas que contengan un ``match'' para la expresión.
    \item Además del código fuente, debe entregar un breve informe conteniendo una descripción de su programa, instrucciones de compilación y ejecución, y ejemplos de uso.
\end{itemize}
\section*{Ejemplo}
Suponga un archivo de texto \texttt{principito} siguiente :\\

\noindent\fbox{%
    \parbox{\textwidth}{%
        Caminando en linea recta no puede uno llegar muy lejos\\
        Fue el tiempo que pasaste con tu rosa lo que la hizo tan importante\\
        Bebo para olvidar que soy un borracho\\
        Lo esencial es invisible para los ojos
    }%
}\\

Entonces :
\begin{lstlisting}[language=bash]
$ cat principito | ./tarea1 "((u.n)|(n.o))"
Caminando en linea recta no puede uno llegar muy lejos
Bebo para olvidar que soy un borracho
$
\end{lstlisting}
En el ejemplo anterior se le entrega a tarea1 el archivo como entrada estándar.
\end{document}
